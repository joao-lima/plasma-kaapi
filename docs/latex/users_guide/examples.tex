%##################################################################################################

\chapter{Use of PLASMA and Examples}

%###################################################################################################
\section{Examples}
In the \texttt{./examples} directory, you will find simple example codes to call
the PLASMA\_[sdcz]gels, PLASMA\_[sdcz]gesv and PLASMA\_[sdcz]posv routines from
a C program or from a Fortran program.  In this section we further explain the necessary
steps for a correct use of these PLASMA routines.

\section{\texttt{PLASMA\_dgesv} example}

Before calling the PLASMA routine \texttt{PLASMA\_dgesv}, some initialization steps 
are required.

Firstly, we set the dimension of the problem $Ax=B$.  In our example, the
coefficient matrix $A$ is 10-by-10, and the right-hand side matrix $B$ 10-by-5.
We also allocate the memory space required for these two matrices.
\begin{verbatim}
in C:
   int N = 10;
   int NRHS = 5;
   int NxN = N*N;
   int NxNRHS = N*NRHS;
   double *A = (double *)malloc(NxN*sizeof(double));
   double *B = (double *)malloc(NxNRHS*sizeof(double));
in Fortran:
      INTEGER           N, NRHS
      PARAMETER         ( N = 10 )
      PARAMETER         ( NRHS = 5 )
      DOUBLE PRECISION  A( N, N ), B( N, NRHS )
\end{verbatim}

Secondly, we initialize the matrix $A$ and $B$ with random values.  Since we
cruelly lack imagination, we use the LAPACK function \texttt{dlarnv} for this
task.  For a starter, you are welcome to change the values in the matrix.
Remember that the interface of \texttt{PLASMA\_dgesv} uses column major format.
\begin{verbatim}
in C:
   int IONE=1;
   int ISEED[4] = {0,0,0,1};   /* initial seed for dlarnv() */
   /* Initialize A */
   dlarnv(&IONE, ISEED, &NxN, A);
   /* Initialize B */
   dlarnv(&IONE, ISEED, &NxNRHS, B);
in Fortran:
      INTEGER           I
      INTEGER           ISEED( 4 )
      EXTERNAL          DLARNV
      DO  I = 1, 4
          ISEED( I ) = 1
      ENDDO
!--   Initialization of the matrix
      CALL DLARNV( 1, ISEED, N*N, A )

!--   Initialization of the RHS
      CALL DLARNV( 1, ISEED, N*NRHS, B )
\end{verbatim}

Thirdly, we initialize PLASMA by calling the \texttt{PLASMA\_Init} routine.
The variable \texttt{cores} specifies the number of cores PLASMA will use.
\begin{verbatim}
in C:
   int cores = 2;
    /* Plasma Initialize */
   INFO = PLASMA_Init(cores);
in Fortran:
      INTEGER     CORES
      PARAMETER   ( CORES = 2 )
      INTEGER     INFO
      EXTERNAL    PLASMA_INIT
! --  Initialize Plasma
      CALL PLASMA_INIT( CORES, INFO )
\end{verbatim}

Before we can call \texttt{PLASMA\_dgesv}, we need to allocate some workspace
necessary for the \texttt{PLASMA\_dgesv} routine to operate.
In PLASMA, each routine has its own routine to allocate its specific workspace arrays. Those
routines are all defined in the \texttt{workspace.c} file. Their names are of
the kind \texttt{PLASMA\_Alloc\_Workspace\_} with the name of the associated
routine in lower case for C and \texttt{PLASMA\_ALLOC\_WORKSPACE\_} with the
name of the associated routine in upper case for Fortran. You will also need to
check the interface since they are different from one routine to the other.
For the \texttt{PLASMA\_dgesv} routine, two arrays need to be initialized,
the workspace (here it is called \texttt{L} in the C code and \texttt{HL} in
the Fortran Code) and the pivots (here it is called \texttt{IPIV} in the C code
and \texttt{HIPIV} in the Fortran Code). Note that in C, you need to use
standard pointers; while in Fortran, you need to use handle defined as an array
of two elements of \texttt{INTEGER*4}.
\begin{verbatim}
in C:
   double *L;
   int *IPIV;
   PLASMA_Alloc_Workspace_dgesv(N, &L, &IPIV);
in Fortran:
      INTEGER*4         HL( 2 ), HPIV( 2 )
      EXTERNAL          PLASMA_ALLOC_WORKSPACE_DGESV
! --  Allocate L and IPIV
      CALL PLASMA_ALLOC_WORKSPACE_DGESV( N, HL, HPIV, INFO )
\end{verbatim}

Finally, 
 we can call the
\texttt{PLASMA\_dgesv} routine. You can report to
the header of the routine for a complete description of the arguments. The
description is also available online in the PLASMA routine description html
page:
\texttt{\url{http://icl.cs.utk.edu/projectsfiles/plasma/html/routine.html}}. As
in LAPACK, PLASMA is returning a return variable, \texttt{INFO}, that indicates if the
exit was successful or not. A successful exit is indicated by \texttt{INFO} equal to \texttt{0}.
In a case of \texttt{INFO} different from \texttt{0}, the value of
\texttt{INFO} should help you to understand the reason of the failure. (See the return
value section in the above cited documentation.)
\begin{verbatim}
in C:
   /* Solve the problem */
   INFO = PLASMA_dgesv(N, NRHS, A, N, L, IPIV, B, N);
   if (INFO < 0)
       printf("-- Error in DGESV example ! \n");
   else
       printf("-- Run successful ! \n");
in Fortran:
! --  Perform the LU solve
      CALL PLASMA_DGESV( N, NRHS, A, N, HL, HPIV, B, N, INFO )
      IF ( INFO < 0 ) THEN
          WRITE(*,*) " -- Error in DGESV example !"
      ELSE
          WRITE(*,*) " -- Run successful !"
      ENDIF

\end{verbatim}

Before exiting the program, we need to free the resource used by
our arrays and finalize PLASMA.
To deallocate the C array, a simple call to \texttt{free} is needed 
whereas in Fortran, PLASMA provides the 
routine \texttt{PLASMA\_DEALLOC\_HANDLE} to do so.
\texttt{PLASMA\_Finalize} call will free all the internal allocated variables and finalize PLASMA.
\begin{verbatim}
in C:
   /* Deallocate L and IPIV */
   free(L); free(IPIV);
   /* Plasma Finalize */
   INFO = PLASMA_Finalize();
in Fortran:
! --  Deallocate L and IPIV
      CALL PLASMA_DEALLOC_HANDLE( HL, INFO )
      CALL PLASMA_DEALLOC_HANDLE( HPIV, INFO )
!--   Finalize Plasma
      CALL PLASMA_FINALIZE( INFO )
\end{verbatim}



