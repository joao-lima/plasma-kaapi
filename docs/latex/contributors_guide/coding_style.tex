%###################################################################################################

\chapter{Coding Style}

%///////////////////////////////////////////////////////////////////////////////////////////////////

\section{FORTRAN Style}
FORTRAN means FORTRAN 77.
Extensions from Fortran 90, Fortran 95, or Fortran 2003 are not allowed.

Currently PLASMA doesn't contain any FORTRAN code in the library. The
only FORTRAN code in PLASMA is located in the
\mbox{\cdl{testing/lin/}} directory. This code is coming from the
Netlib LAPACK testings and differ only
by the call to the equivalent PLASMA routines in place of LAPACK ones.
The advantage of keeping this code in FORTRAN is its close resemblance of
LAPACK code, from which the code is derived.
By the same token, the main coding rule, applying to the development and maintenance
of this code, is that it should follow LAPACK as closely as possible.
This applies to the use of whitespaces, punctuation, indentation, line breaking,
the use of lower and uppercase characters, comments, variable naming, etc.

%///////////////////////////////////////////////////////////////////////////////////////////////////

\section{C Style}

Only code that conforms to the ANSI~C standard is allowed. The standard
is commonly referred to as C89 and was ratified by ISO.
One way to check for compliance is to use the following command:
\begin{cmd_line}
\cdl{gcc -std=c89 -W -Wall -pedantic -c plasma.c}
\end{cmd_line}
Since the C89 standard does not support complex data types the
following command needs to be used to remove warnings about it:
\begin{cmd_line}
\cdl{gcc -std=c99 -W -Wall -pedantic -c plasma.c}
\end{cmd_line}
PLASMA code needs to be portable and work on Windows where the most commonly
used compiler is a C++ compiler. PLASMA code must then compile with a C++
compiler. The following command will compile a C source code using
the GNU C++ compiler:
\begin{cmd_line}
\cdl{gcc -x c++ -W -Wall -pedantic -c plasma.c}
\end{cmd_line}

\begin{description}
\item[No Trailing Whitespaces:]
There should be no trailing whitespace characters at the end of lines,
no whitespace characters in empty lines and no whitespace characters
at the end of files (The last closing curly bracket should be followed
by a single newline). This is easy to accomplish by using an editor that
shows whitespace characters, such as Kwrite, Kate, Emacs~(just use
\cdl{M-x delete-trailing-whitespace} command). Otherwise a \cdl{sed},
\cdl{awk}, or \cdl{perl} ``one-liner'' script can be used to clean up
the file before committing to the repository (e.g., \cdl{tools/code\_cleanup}).

\item[Whitespace Separators:]
There should be a whitespace between a C language keyword and the left round
bracket and a whitespace between the right round bracket and the left curly
bracket. There should be no whitespace immediately after left round bracket
and immediately before right round bracket.
Comas separating arguments are followed by a single space and not preceded
by a space.

\item[End-of-line Management:]
Every file should have an end-of-line character at the end unless it's a
zero-length file. End-of-file character is \cdl{\textbackslash{n}} (as it
is on Unix including Linux; ASCII code $10$). Other end-of-line schemes should not be
used: Windows and DOS (\cdl{\textbackslash{n}\textbackslash{r}} -- ASCII
codes $10$ and $13$) and Mac (\cdl{\textbackslash{r}} -- ASCII code
$13$).

\item[Indentation:]
The unit of indentation is four spaces. The left curly bracket follows the
control flow statement in the same line. There is no newline between the
control flow statement and the block enclosed by curly braces. The closing
curly bracket is in a new line right after the end of the enclosed block.

There is no specific limit on the length of lines. Up to a 100 columns is
fine. Clarity is paramount. For multi-line function calls it is recommended
that new lines start in the column immediately following the left bracket.

\item[Tabs:]
Tab characters should not be used. Tabs should always be emulated by four
spaces, a feature available in almost any text editor. If that proves
difficult, again, a \cdl{sed}, \cdl{awk}, or \cdl{perl} ``one-liner'' can be used to do the replacement
before the commit.

\item[Variable Declarations:]
For the most part all variables should be declared at the beginning of each
function, unless doing otherwise significantly improves code clarity in
a specific case.

\item[Constants:]
Constants should have appropriate types. If a constant serves as a floating
point constant, it should be written with the decimal point.
If a constant is a bit mask, it is recommended that it is given in
hexadecimal notation.

\item[printf Strings:]
ANSI C concatenates strings separated by whitespace. There is no need for
multiple \cdl{printf} calls to print a multi-line message. One \cdl{printf}
can be used with multiple strings.

\item[F77 Trailing Underscore:]
When calling a FORTRAN function the trailing underscore should never be used.
If the underscore is needed it should be added by an appropriate conditional
preprocessor definition in an appropriate header file (e.g.: \cdl{core\_blas.h},
\cdl{lapack.h}).

\item[Special Characters:]
No special characters should be used used in the code. The ASCII
codes allowed in the file are between $32$ and $127$ and code $10$
for new line.
\end{description}

%///////////////////////////////////////////////////////////////////////////////////////////////////

\section{Coding Practices}


\begin{description}
\item[Preprocessor Macros:]
Conditional compilation, through the {\em \#define} directive,
should only be used for portability reasons
and never for making choices that can be decided at runtime.
Excessive use of the {\em \#define} macros leads to frequent recompilations and obscure code.

\item[Dead Code:]
There should be no dead code: no code that is never executed, no including
of header files that are not necessary, no unused variables. Dead code can
be justified if it serves as a comment, e.g., canonical form of optimized
code. In such case the code should be in comments.

\item[OS Interactions:]
Error checks have to follow each interaction with the OS.
The code should never be terminated by the OS.
In particular each memory allocation should be checked.
The code cannot produce a segmentation fault.

\item[User Interactions:]
User input needs to be checked for correctness.
The user should not be able to cause undefined behavior.
In particular the user should not be able to cause termination of the code
by the OS.
\end{description}

%///////////////////////////////////////////////////////////////////////////////////////////////////

\section{Naming Convention}

Any externally visible C symbols should be prefixed with \cdl{PLASMA\_}.
Following the prefix, the name should be in lower case~(this will create
a mixed-case name and thus will guarantee the lack of name clashes with FORTRAN
interfaces that are always either all lower-case or all upper-case).
For example: \cdl{PLASMA\_dgetrf}. This is in line with C interfaces
for MPI~(\cdl{MPI\_Send}), PETSc, and BLAS~(\cdl{BLAS\_dgemm}).

%///////////////////////////////////////////////////////////////////////////////////////////////////

\section{Boiler Plate text/code for Each File}
Copyright, License, year, \ldots

%///////////////////////////////////////////////////////////////////////////////////////////////////

\section{Exceptions}
As often is the case all rules have exceptions. Exceptions should only be
used after consulting with the PLASMA team members.

%###################################################################################################
